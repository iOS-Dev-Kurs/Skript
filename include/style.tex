
% General Packages
\usepackage[ngerman]{babel}
\usepackage[xetex]{graphicx}
\usepackage{enumitem}
\usepackage{hyperref}


% Font
\usepackage{fontspec,xunicode}
\setmainfont{Palatino}
\setsansfont{Optima}
\setmonofont[Scale=MatchLowercase]{Menlo} % MatchLowercase is pretty cool


% PDF Meta Info
\makeatletter
\AtBeginDocument{
  \hypersetup{
    pdftitle = {\@title{} \@subtitle},
    pdfauthor = {\@author}
  }
}
\makeatother


% Titlepage
\renewcommand{\maketitle}{
    \input{include/titlepage}
}


% Page Layout
%\pagestyle{headings}
\usepackage[textheight=640pt]{geometry}
\usepackage{fancyhdr}
\pagestyle{fancy}
\AtBeginDocument{
\fancyhf{}
\lhead{\textit{Nils Fischer, Universität Heidelberg, \shortsemester}}
\rhead{\textit{\shortdoctype{} \nouppercase{\leftmark}}}
\renewcommand{\headrulewidth}{0.2pt}
\cfoot{\thepage}
}


% Headings
\usepackage[bf]{titlesec}
%\titleformat{\chapter}[display]{\large\bfseries}{\chaptertitlename\ \thechapter}{5pt}{\large}% NEW
%\titlespacing*{\chapter}{0pt}{30pt}{20pt}% NEW


% Menukeys
\usepackage{menukeys}


% Style

\newcommand\strong\textbf

\usepackage{color,xcolor}
\definecolor{string}{RGB}{196,26,22}
\newcommand\str[1]{\textcolor{string}{\texttt{"#1"}}}

\newcommand\filename[1]{\str{#1}}
\newenvironment{filecontent}[1]{\textbf{#1}}{}


% References
\newcommand{\linkref}[1]{[\footnote{\url{#1}}]}
\newcommand{\abbref}[1]{\emph{(\hyperref[#1]{s. S. \pageref{#1}, Abb. \ref{#1}})}}
\newcommand{\secref}[1]{\emph{(\hyperref[#1]{s. S. \pageref{#1}, Abschnitt \ref{#1}})}}
\newcommand{\excref}[1]{\emph{(\hyperref[#1]{s. S. \pageref{#1}, Übungsaufgabe \ref{#1}})}}
\newcommand{\skriptref}[1]{\emph{Relevante Kapitel im Skript: #1}}


% Images
\newcommand{\includegraphicsc}[4][\textwidth]{{\begin{figure}[ht]\centering\includegraphics[width=#1]{#2}\caption{#4}\label{#3}\end{figure}}}
\newcommand{\screenshotwidth}{.8\textwidth}
\newcommand{\iphonewidth}{.4\textwidth}


% Margin Images
\usepackage{marginnote}
\newcommand{\mvcindicatormodel}{\reversemarginpar\marginnote{\centering
    \includegraphics[width=1cm]{img/mvc_model}\\
    \textbf{\small Model}
}}
\newcommand{\mvcindicatorview}{\reversemarginpar\marginnote{\centering
    \includegraphics[width=1cm]{img/mvc_view}\\
    \textbf{\small View}
}}
\newcommand{\mvcindicatorcontroller}{\reversemarginpar\marginnote{\centering
    \includegraphics[width=1cm]{img/mvc_controller}\\
    \textbf{\small Controller}
}}


% Lecture structure
\newenvironment{lecture}{}{\pagebreak}


% Exercise Style
\newcounter{exc}
\newenvironment{exc}{\clearpage\subsubsection{Übungsaufgaben}\begin{enumerate}\setcounter{enumi}{\theexc}}{\setcounter{exc}{\theenumi}\end{enumerate}}

\newenvironment{excitem}[3]{\item\label{exc:#1}\addcontentsline{toc}{section}{\emph{Übungsaufgabe:} #2}\strong{#2}\hfill [#3 P.]}{}
\newenvironment{excitem*}[3]{\begin{excitem}{#1}{Optional: #2}{+#3}}{\end{excitem}}

\newcommand{\excextra}[1]{\strong{Extra:} #1}
\newcommand{\exchinweis}[1]{\strong{Hinweis:} #1}
\newenvironment{hinweis}{\strong{Hinweis:} }{}
\newenvironment{exchinweise}{\strong{Hinweise:}\begin{itemize}}{\end{itemize}}
%\newenvironment{lsg}{\begin{enumerate}}{\end{enumerate}}
\newenvironment{lsg}{}{}
%\newenvironment{lsgitem}[2]{\item\label{lsg:#1}\addcontentsline{toc}{chapter}{\theenumi. #2}\strong{#2}}{}
\newenvironment{lsgitem}[2]{\label{lsg:#1}\section{#2}}{}
