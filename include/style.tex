\usepackage[ngerman]{babel}

\usepackage{fontspec}
\setmainfont{Cardo}
\usepackage{xunicode}

\usepackage{graphicx}
\usepackage{enumitem}
\usepackage[pdftex,
            pdfauthor={Nils Fischer},
            pdftitle={Softwareentwicklung für iOS mit Objective-C und Xcode}]{hyperref}

\newcommand\emphc\textbf
\newcommand{\linkref}[1]{[\footnote{\url{#1}}]}
\newcommand{\abbref}[1]{\emph{(\hyperref[#1]{s. S. \pageref{#1}, Abb. \ref{#1}})}}
\newcommand{\secref}[1]{\emph{(\hyperref[#1]{s. S. \pageref{#1}, Abschnitt \ref{#1}})}}
\newcommand{\excref}[1]{\emph{(\hyperref[#1]{s. S. \pageref{#1}, Übungsaufgabe \ref{#1}})}}
\newcommand{\skriptref}[1]{\emph{Relevante Kapitel im Skript: #1}}

\newcommand{\includegraphicsc}[4][\textwidth]{{\begin{figure}[ht]\centering\includegraphics[width=#1]{#2}\caption{#4}\label{#3}\end{figure}}}
\newcommand{\screenshotwidth}{.8\textwidth}
\newcommand{\iphonewidth}{.4\textwidth}

\newcommand\exctitle\emphc
\newenvironment{exc}{\subsubsection{Übungsaufgaben}}{}
\newenvironment{exclsg}{\pagebreak\subsubsection{Lösung der Übungsaufgaben}}{\pagebreak}

\usepackage{menukeys}
\newcommand{\keysc}{\keys}
\newcommand{\menuc}{\menu}
\newcommand\cmdkey{{\cmd}}
\newcommand\shiftkey{{\shift}}
\newcommand\altkey{{\Alt}}
\newcommand\ctrlkey{{\ctrl}}

\pagestyle{headings}